\subsubsection{Random Forest Performance}
Random Forest Classifiers do not require separable data to find meaningful patterns, as demonstrated. However, noise reduction resulted in significantly better results. \\ 

It should however be noted that while Random Forest achieved a 76.9\% accuracy on the temporally smoothed data, as seen in Figure~\ref{fig:rfTemporalSmoothing}, the model performed exceedingly well at classifying label 1 (eyes closed), whereas it achieved only a 60\% accuracy for label 0 (eyes open). Interestingly, the classes were balanced in the training data, and the imbalance was only present in the test data (Table~\ref{tab:smoothedDistribution}). This suggests that the class imbalance was not the reason for this trouble generalising; more likely, this has to do with the strength of the signals. 
\begin{table}[H]
    \centering
    \begin{tabular}{|c||c|c|}
        \hline 
        Set & Label 0 (eyes open) & Label 1 (eyes closed) \\ \hline \hline
        Train & 63 (47\%) & 70 (53\%) \\ \hline 
        Test & 40 (71\%) & 16 (29\%)\\ \hline 
    \end{tabular}
    \caption{Label distribution for temporally smoothed test data}
    \label{tab:smoothedDistribution}
\end{table}
