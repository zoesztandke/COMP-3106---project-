\subsubsection{Support Vector Machine}
A Support Vector Machine (SVM) is a supervised classification algorithm that works to maximise the distance between a hyperplane and the nearest data points from each class. It takes a regularisation parameter \(C\) and a kernel coefficient, gamma (\(\gamma\)). A Radial Basis Function (RBF) kernel was used, as it was the optimal choice for non-linear data. 

SVM performance depends heavily on how separable the classes are. Due to the PCA results showing significant overlap between the different classes, the SVM was expected to perform poorly, and this was reflected in the results.